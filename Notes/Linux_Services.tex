\documentclass[10pt, a4paper, onecolumn, openany]{book} % openany make chapter start whenever, DELETE in OFFICIAL

% PACKAGES:
% Font Encoding
\usepackage[utf8]{inputenc}     % Use UTF-8
\usepackage[T1]{fontenc}        % T1 font encoding (latin characters)
% Header
\usepackage{fancyhdr}           % fancy page header options
\usepackage{titlesec}           % used to have \thechapter in same line as \chaptertitlename
% MISC
\usepackage{hyperref}           % \url{}
\usepackage{xurl}
\usepackage{graphicx}           % images
\usepackage{xcolor}            % colors
\usepackage{fancyvrb}          % colors in Verbatin, header: \begin{Verbatim}[commandchars=\\\{\}]

% DECORATIVE LINES + CHAPTER IN SAME LINE:
\renewcommand{\headrulewidth}{2pt}  % Top decorative line
\renewcommand{\footrulewidth}{2pt}  % Bottom decorative line
\pagestyle{fancy}                   % better header for normal pages, not only chapter ones
\fancyhf{}                          % clear header and adjust as wanted:
    \chead{\leftmark}               % header
    \cfoot{Page \thepage}           % footer
\fancypagestyle{plain}{
\fancyhf{} 
    \chead{\leftmark}       % header
    \cfoot{Page \thepage}   % footer
}
\renewcommand{\chaptername}{}       % change word chapter to {}
\titleformat{\chapter}[hang]{\normalfont\huge\bfseries}{\chaptertitlename\ \thechapter.}{1em}{} % Chapter in same line as chapter name

% SIZES OF SECTIONS:
\titleformat*{\section}{\LARGE\bfseries}
\titleformat*{\subsection}{\Large\bfseries}
\titleformat*{\subsubsection}{\large\bfseries}

% DISABLE huge space after (paragraph indent) section name before text starts:
\setlength{\parindent}{0pt}

% COLOR EXAMPLES:
% \definecolor{MyColor}{RGB}{219, 48, 122}  % define
% \textcolor{MyColor}{Some random text}     % usage in document
\definecolor{root}{RGB}{222, 0, 0}
\definecolor{user}{RGB}{0, 150, 00}
\definecolor{dir}{RGB}{0, 100, 200}
\definecolor{file}{RGB}{77, 187, 101}
\definecolor{block}{RGB}{255, 80, 0}
\definecolor{command}{RGB}{41, 182, 0}
\definecolor{comment}{RGB}{0, 182, 182}
\definecolor{background}{RGB}{240, 240, 240}
\definecolor{nocommand}{RGB}{222, 0, 0}
\definecolor{showcommand}{RGB}{255, 80, 0}

% COLORS FOR CODE in document:
%\begin{minted}[frame=lines,framesep=2mm,baselinestretch=1.2,fontsize=\footnotesize,linenos]{js}
%\end{minted}

% IMAGES:
\graphicspath{./images/} % define directory
% \includegraphics[scale=1.5]{./images/random_image.png} % usage in document

% TABLE:
% \begin{center}
%    \begin{small}
%    \begin{tabular}{|p{1cm}|p{1cm}|p{1cm}|p{1cm}|p{1cm}|p{1cm}|p{1cm}|}
%    \hline
%    + & - & * & / & \% & ** & () \\
%    \hline
%    1 & 2 & 3 & 4 & 5 & 6 & 7 \\
%    \hline
%    \end{tabular}
%    \end{small}
%\end{center}

%\titlespacing*{\section}{0pt}{1.5cm}{0.2cm}
%\titlespacing*{\subsection}{0pt}{0.2cm}{0.2cm}

% colored verbatim BG:
%\let\oldv\verbatim
%\let\oldendv\endverbatim
%\def\verbatim{\par\setbox0\vbox\bgroup\oldv}
%\def\endverbatim{\oldendv\egroup\fboxsep0pt \noindent\colorbox[gray]{0.9}{\usebox0}\par}
% colored Verbatim BG (supported colors: https://linuxhint.com/change-text-colors-latex/):
\renewcommand{\FancyVerbFormatLine}[1]{\colorbox{background}{#1}}

% Section starts on new page:
%\newcommand{\sectionbreak}{\clearpage}



%%%%%%%%%%%%%%%%%%%%%%%%%%%%%%%%%%%%%%%%%%%%%%%%%%%%%%%%%%%%%%%%%%%%%%%%%%%%
%%%%%%%%%%%%%%%%%%%%%%%%%%%%%%%%% TITLE %%%%%%%%%%%%%%%%%%%%%%%%%%%%%%%%%%%%
%%%%%%%%%%%%%%%%%%%%%%%%%%%%%%%%%%%%%%%%%%%%%%%%%%%%%%%%%%%%%%%%%%%%%%%%%%%%
\title{\textbf{Linux Services}}
\author{AISK}
\date{November, 2021}
%%%%%%%%%%%%%%%%%%%%%%%%%%%%%%%%%%%%%%%%%%%%%%%%%%%%%%%%%%%%%%%%%%%%%%%%%%%%
%%%%%%%%%%%%%%%%%%%%%%%%%%%%%%%%% START %%%%%%%%%%%%%%%%%%%%%%%%%%%%%%%%%%%%
%%%%%%%%%%%%%%%%%%%%%%%%%%%%%%%%%%%%%%%%%%%%%%%%%%%%%%%%%%%%%%%%%%%%%%%%%%%%
\begin{document}
\maketitle
%\clearpage % official blank page
\tableofcontents


% DHCP, DNS, IPTABLES...



%%%%%%%%%%%%%%%%%%%%%%%%%%%%%%%%%%%%%%%%%%%%%%%%%%%%%%%%%%%%%%%%%%%%%%%%%%%%
%%%%%%%%%%%%%%%%%%%%%%%%%%%%%%%%% SSH %%%%%%%%%%%%%%%%%%%%%%%%%%%%%%%%%%%%
%%%%%%%%%%%%%%%%%%%%%%%%%%%%%%%%%%%%%%%%%%%%%%%%%%%%%%%%%%%%%%%%%%%%%%%%%%%%
\chapter{SSH}
%%%%%%%%%%%%%%%%%%%%%%%%%%%%%%%%%%%%%%%%%%%%%%%%%%%%%%%%%%%%%%%%%%%%%%%%%%%%
\section{SSH Audit}
\begin{enumerate}
    \item \textbf{Dependencies:}
\begin{Verbatim}[commandchars=\\\{\}]
\textcolor{root}{root#} \textcolor{command}{apt} install ssh-audit [-y]
\end{Verbatim}   
    \item \textbf{Usage:}
\begin{Verbatim}[commandchars=\\\{\}]
\textcolor{user}{user\$} \textcolor{command}{ssh-audit} [-p <PORT>] <HOST>
\end{Verbatim}       
\end{enumerate}
%%%%%%%%%%%%%%%%%%%%%%%%%%%%%%%%%%%%%%%%%%%%%%%%%%%%%%%%%%%%%%%%%%%%%%%%%%%%
\section{Client}
\begin{enumerate}
    \item \textbf{Dependencies:}
\begin{Verbatim}[commandchars=\\\{\}]
\textcolor{root}{root#} \textcolor{command}{apt} install openssh-client [-y]
\end{Verbatim}
    \item \textbf{Connect:}
\begin{Verbatim}[commandchars=\\\{\}]
\textcolor{root}{root#} \textcolor{command}{ssh} [-p <PORT>] <USER>@<HOST> [-c <3des-cbc>] 
[-oKexAlgorithms=+<diffie-hellman-group1-sha1>]
\end{Verbatim}
    \item \textbf{SCP}
    \begin{itemize}
        \item \textbf{GET:}
\begin{Verbatim}[commandchars=\\\{\}]
\textcolor{root}{root#} \textcolor{command}{scp} [-r] [-P <PORT>] <USER>@<HOST>:</home/USER/DIR/> <.>
\end{Verbatim}
        \item \textbf{PUT:}
\begin{Verbatim}[commandchars=\\\{\}]
\textcolor{root}{root#} \textcolor{command}{scp} [-r] [-P <PORT>] <.> <USER>@<HOST>:</home/USER/DIR/>
\end{Verbatim}
    \end{itemize}
\end{enumerate}
%%%%%%%%%%%%%%%%%%%%%%%%%%%%%%%%%%%%%%%%%%%%%%%%%%%%%%%%%%%%%%%%%%%%%%%%%%%%
\section{Server}
\subsection{Configuration}
\begin{enumerate}
    \item \textbf{Dependencies:}
\begin{Verbatim}[commandchars=\\\{\}]
\textcolor{root}{root#} \textcolor{command}{apt} install openssh-server [-y]
\end{Verbatim}
    \item \textbf{Configure SSHD:}
    \newline File (\textbf{\textcolor{file}{/etc/ssh/sshd\_config}}):
\newline \underline{\url{https://github.com/AISK11/debian/blob/main/config_files/ssh/sshd_config}}
    \item \textbf{Start SSH on startup:}
\begin{Verbatim}[commandchars=\\\{\}]
\textcolor{root}{root#} \textcolor{command}{systemctl} <enable|disable> ssh.service
\end{Verbatim}
\end{enumerate}
\subsection{Logging}
\begin{itemize}
    \item \textbf{Default file:} (\textbf{\textcolor{file}{/var/log/auth.log}}).
\end{itemize}
\begin{enumerate}
    \item \textbf{Create rsyslog rule:}
\newline File: (\textbf{\textcolor{file}{/etc/rsyslog.d/sshd.conf}}):
\begin{verbatim}
if $programname == 'sshd' then /var/log/sshd.log
\end{verbatim}
    \item \textbf{Create logrotate rule:}
\newline File: (\textbf{\textcolor{file}{/etc/logrotate.d/sshd}}):
\begin{verbatim}
/var/log/sshd.log
{
    missingok
    notifempty
    rotate 4
    weekly
    create 0600 root root
    compress
    delaycompress
    copytruncate
    nomail
    shred
}
\end{verbatim}
    \item \textbf{Add crontab rules:}
\newline File: (\textbf{\textcolor{file}{/etc/logrotate.d/sshd}}):
\begin{Verbatim}[commandchars=\\\{\}]
\textcolor{root}{root#} \textcolor{command}{echo} -e "@reboot systemctl restart logrotate.service 
&& systemctl restart rsyslog.service" >> \textcolor{file}{/var/spool/cron/crontabs/root}
\textcolor{root}{root#} \textcolor{command}{echo} -e "@daily systemctl restart logrotate.service 
&& systemctl restart rsyslog.service" >> \textcolor{file}{/var/spool/cron/crontabs/root}
\end{Verbatim}
\end{enumerate}
%%%%%%%%%%%%%%%%%%%%%%%%%%%%%%%%%%%%%%%%%%%%%%%%%%%%%%%%%%%%%%%%%%%%%%%%%%%%
%%%%%%%%%%%%%%%%%%%%%%%%%%%%%%%%% Endlessh %%%%%%%%%%%%%%%%%%%%%%%%%%%%%%%%%%%%
%%%%%%%%%%%%%%%%%%%%%%%%%%%%%%%%%%%%%%%%%%%%%%%%%%%%%%%%%%%%%%%%%%%%%%%%%%%%
\chapter{Endlessh}
%%%%%%%%%%%%%%%%%%%%%%%%%%%%%%%%%%%%%%%%%%%%%%%%%%%%%%%%%%%%%%%%%%%%%%%%%%%%
\section{Configuration}
\begin{enumerate}
    \item \textbf{Dependencies:}
\begin{Verbatim}[commandchars=\\\{\}]
\textcolor{root}{root#} \textcolor{command}{apt} install endlessh [-y]
\end{Verbatim}
    \item \textbf{Configure Endlessh:}
\newline File (\textbf{\textcolor{file}{/etc/endlessh/config}}).
\newline \underline{\url{https://github.com/AISK11/debian/blob/main/config_files/Honeypot/endlessh/config}}
    \item \textbf{Start Endlessh on startup:}
\begin{Verbatim}[commandchars=\\\{\}]
\textcolor{root}{root#} \textcolor{command}{systemctl} enable endlessh.service
\end{Verbatim}
\end{enumerate}
%%%%%%%%%%%%%%%%%%%%%%%%%%%%%%%%%%%%%%%%%%%%%%%%%%%%%%%%%%%%%%%%%%%%%%%%%%%%
\section{Logging}
\begin{itemize}
    \item \textbf{Default file:} (\textbf{\textcolor{file}{/etc/var/syslog}}).
\end{itemize}
\begin{enumerate}
    \item \textbf{Create rsyslog rule:}
\newline File: (\textbf{\textcolor{file}{/etc/rsyslog.d/endlessh.conf}}):
\begin{verbatim}
if $programname == 'endlessh' then /var/log/endlessh.log
\end{verbatim}
    \item \textbf{Create logrotate rule:}
\newline File: (\textbf{\textcolor{file}{/etc/logrotate.d/endlessh}}):
\begin{verbatim}
/var/log/endlessh.log
{
    missingok
    notifempty
    rotate 4
    weekly
    create 0600 root root
    compress
    delaycompress
    copytruncate
    nomail
    shred
}
\end{verbatim}
    \item \textbf{Add crontab rules:}
\newline File: (\textbf{\textcolor{file}{/etc/logrotate.d/sshd}}):
\begin{Verbatim}[commandchars=\\\{\}]
\textcolor{root}{root#} \textcolor{command}{echo} -e "@reboot systemctl restart logrotate.service 
&& systemctl restart rsyslog.service" >> \textcolor{file}{/var/spool/cron/crontabs/root}
\textcolor{root}{root#} \textcolor{command}{echo} -e "@daily systemctl restart logrotate.service 
&& systemctl restart rsyslog.service" >> \textcolor{file}{/var/spool/cron/crontabs/root}
\end{Verbatim}
\end{enumerate}
%%%%%%%%%%%%%%%%%%%%%%%%%%%%%%%%%%%%%%%%%%%%%%%%%%%%%%%%%%%%%%%%%%%%%%%%%%%%
%%%%%%%%%%%%%%%%%%%%%%%%%%%%%%%%% VNC %%%%%%%%%%%%%%%%%%%%%%%%%%%%%%%%%%%%
%%%%%%%%%%%%%%%%%%%%%%%%%%%%%%%%%%%%%%%%%%%%%%%%%%%%%%%%%%%%%%%%%%%%%%%%%%%%
\chapter{VNC}
%%%%%%%%%%%%%%%%%%%%%%%%%%%%%%%%%%%%%%%%%%%%%%%%%%%%%%%%%%%%%%%%%%%%%%%%%%%%
\section{Client}
\begin{enumerate}
    \item \textbf{Dependencies:}
\begin{Verbatim}[commandchars=\\\{\}]
\textcolor{root}{root#} \textcolor{command}{apt} install tigervnc-viewer [-y]
\end{Verbatim}    
    \item \textbf{Connect:}
    \begin{itemize}
        \item \textbf{Direct Access:}
\begin{Verbatim}[commandchars=\\\{\}]
\textcolor{user}{user\$} \textcolor{command}{xtigervncviewer} <HOST>::<5900>
\end{Verbatim}   
    \item \textbf{SSH Tunneling:}
    \begin{enumerate}
        \item \textbf{Create SSH Tunel:}
\begin{Verbatim}[commandchars=\\\{\}]
\textcolor{user}{user\$} \textcolor{command}{ssh} [-p <PORT>] -L <L_PORT>:<127.0.0.1>:<5900> -N <USER>@<HOST>
\end{Verbatim}          
        \item \textbf{Connect:}
\begin{Verbatim}[commandchars=\\\{\}]
\textcolor{user}{user\$} \textcolor{command}{xtigervncviewer} localhost::<L_PORT>
\end{Verbatim}  
    \end{enumerate}
    \end{itemize}
    \item \textbf{Context GUI Menu:}
    \newline "F8"
\end{enumerate}
%%%%%%%%%%%%%%%%%%%%%%%%%%%%%%%%%%%%%%%%%%%%%%%%%%%%%%%%%%%%%%%%%%%%%%%%%%%%
\section{Server}
%%%%%%%%%%%%%%%%%%%%%%%%%%%%%%%%%%%%%%%%%%%%%%%%%%%%%%%%%%%%%%%%%%%%%%%%%%%%
\subsection{Virtual (standalone) Server}
\begin{enumerate}
    \item \textbf{Dependencies:}
\begin{Verbatim}[commandchars=\\\{\}]
\textcolor{root}{root#} \textcolor{command}{apt} install tigervnc-standalone-server [-y]
\end{Verbatim}
\end{enumerate}

%%%%%%%%%%%%%%%%%%%%%%%%%%%%%%%%%%%%%%%%%%%%%%%%%%%%%%%%%%%%%%%%%%%%%%%%%%%%
\subsection{Current display Server}
\begin{enumerate}
\begin{Verbatim}[commandchars=\\\{\}]
\textcolor{root}{root#} \textcolor{command}{apt} install tigervnc-scraping-server [-y]
\end{Verbatim}
\end{enumerate}



\end{document}
